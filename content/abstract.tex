\null\vfil
\begin{center}\textsf{\textbf{\abstractname}}\end{center}

\iflanguage{spanish}{
    \noindent Actualmente los chatbots representan una estrategia de negocio novedosa en la atención a clientes y optimización de procesos en los que se implementan interacciones cliente-empresa, por lo que su estudio así como implementación y mejora representan una gran oportunidad de desarrollo de software. Desafortunadamente, varias herramientas relacionadas con la implementación de chatbots se hallan bajo licencias privativas, por lo que la comunidad open-source requiere de una implementación fácil de implementar así como útil. En el presente trabajo se abordará el diseño y la implementación de un conector open source entre los proyectos Asterisk y Rasa, en el cual se involucra el procesamiento de lenguaje natural, la telefonía IP y una respuesta de voz interactiva mediante un chatbot. 
}{
    \noindent Currently, chatbots represent a novel business strategy in customer service and process optimization in which customer-company interactions are implemented, so their study as well as implementation and improvement represent a great software development opportunity. Unfortunately, several tools related to the implementation of chatbots are under proprietary licenses, so the open-source community requires an implementation that is easy to implement as well as useful. This work will address the design and implementation of an open source connector between the Asterisk and Rasa projects, which involves natural language processing, IP telephony and an interactive voice response through a chatbot.
}
\vfil\null
