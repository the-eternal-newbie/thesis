% => Laut Aussage des Studienreferats braucht es - auch wenn die Arbeit in englischer Sprache verfasst ist - KEINE separate Version der Eigenstandigkeitserklarung auf Englisch. Sowohl für Arbeiten in deutscher Sprache als auch für Arbeiten in englischer Sprache genügt EINE EINZIGE Eigenstandigkeitserklarung auf DEUTSCH.
\begin{otherlanguage}{ngerman}

    \begin{center}\textsf{\textbf{Eidesstattliche Erklarung}}\end{center}
    Hiermit versichere ich, dass meine {\hpitype} \enquote{\hpititle} (\enquote{\hpititleother}) selbstandig verfasst wurde und dass keine anderen Quellen und Hilfsmittel als die angegebenen benutzt wurden. Diese Aussage trifft auch für alle Implementierungen und Dokumentationen im Rahmen dieses Projektes zu.\\
    
    \noindent
    Potsdam, den \hpidate,
    \vspace{2cm}
    
    \begin{center}
    \begin{tabular}{C{6cm}}
    \hline
    {\small({\hpiauthor})}
    \end{tabular}
    \end{center}
    
    \end{otherlanguage}
    
    
    