\chapter{\iflanguage{spanish}{Introducción}{Introduction}}

Los “chatbots” se están convirtiendo en una de las principales herramientas para las empresas, principalmente para tareas de atención al cliente y consultas de información.
Entre las herramientas para crear uno, se encuentra Rasa[1], de código abierto[2], multiplataforma al estar escrita en Python y con constantes actualizaciones de los desarrolladores. Estas características la convierten en una de las preferidas.
Uno de los temas que más interés ha generado en la comunidad últimamente es la comunicación por medio de voz con el chatbot. Ya existen conectores para interfaces de páginas web. La otra área que representa un gran interés es la telefonía, por lo que hace falta una manera de usar chatbots en este ámbito.
Para la integración con telefonía se cuenta con Asterisk [3], también de código abierto[4], una de las herramientas más extendidas en el mundo e integrada en proyectos de comunicaciones como FreePBX y Elastix.